\documentclass{article}

% Packages (optional, for more features)
\usepackage[utf8]{inputenc}
\usepackage{fontspec}  
\usepackage{polyglossia} 
\usepackage{amsmath} % for math
\usepackage{graphicx} % for images
\usepackage{nicematrix}
\usepackage{tikz}
\usetikzlibrary{fit, shapes}
\setmainlanguage{russian} 
\setotherlanguage{english}
\setmainfont{Times New Roman}

\title{Карты карно}

\author{Тарасенко Алексей Романович}

\date{\today}

\begin{document}

\maketitle

\section{Карты карно}

Сами по себе карты карно представляют собой матрицу, построенную по уже заранее определённому шаблону. Эту матрицу необходимо заполнить в соответствии с заданной логической функцией. Образцы карт карно для 2-х и 3-х перемнных соответственно:

\subsection{Образцы}

\begin{tabular}{c|c|c}
$x_{1} \backslash x_{2}$ & 0 & 1 \\
\hline
0 & f(0,0) & f(0,1) \\
1 & f(1,0) & f(1,1) \\
\end{tabular}


\begin{tabular}{c|c|c|c|c}
$x_{1} \backslash x_{2}x_{3} $ & 00 & 01 & 11 & 10 \\
\hline
0 & f(0,0,0) & f(0,0,1) & f(0,1,1) & f(0,1,0) \\
1 & f(1,0,0) & f(1,0,1) & f(1,1,1) & f(1,1,0) \\
\end{tabular}

\subsection{Примеры}

Для примеры я возьму элементарные функции выраженные через СДНФ и СКНФ.

Пример 1: $\overline{x_{1}} \land x_{2} \lor x_{1} \land \overline{x_{2}} $

Заполненная карта карно:


\begin{NiceTabular}{c|c}[colortbl-like]
$x_{1}$ & $x_{2}$ \\
\hline
0 & 1 \\
0 & 1 \\
\CodeAfter
\tikz \node[draw=red, thick, ellipse, fit=(2-2)(3-2)] {};
\end{NiceTabular}

\begin{tabular}{c|c|c|c|c}
$x_{1} \backslash x_{2}x_{3} $ & 00 & 01 & 11 & 10 \\
\hline
0 & 1 & 1 & 0 & 0 \\
1 & 1 & 0 & 0 & 1 \\
\end{tabular}

\section{Conclusion}

That's all for now. You can add more sections, equations, or figures as needed.

\end{document}
