\documentclass{article}

% Packages (optional, for more features)
\usepackage[utf8]{inputenc}
\usepackage{fontspec}  
\usepackage{polyglossia} 
\usepackage{nicematrix}
\usepackage{amsmath} % for math
\usepackage{graphicx} % for images
\usepackage{tikz}
\usetikzlibrary{fit, shapes}
\setmainlanguage{russian} 
\setotherlanguage{english}
\setmainfont{Times New Roman}

\title{Карты карно}

\author{Тарасенко Алексей Романович}

\date{\today}

\begin{document}

\maketitle

\section{Карты карно}

Сами по себе карты карно представляют собой матрицу, построенную по уже заранее определённому шаблону. Эту матрицу необходимо заполнить в соответствии с заданной логической функцией. Образцы карт карно для 2-х и 3-х перемнных соответственно:

\subsection{Образцы}

\begin{tabular}{c|c|c}
$x_{1} \backslash x_{2}$ & 0 & 1 \\
\hline
0 & f(0,0) & f(0,1) \\
1 & f(1,0) & f(1,1) \\
\end{tabular}


\begin{tabular}{c|c|c|c|c}
$x_{1} \backslash x_{2}x_{3} $ & 00 & 01 & 11 & 10 \\
\hline
0 & f(0,0,0) & f(0,0,1) & f(0,1,1) & f(0,1,0) \\
1 & f(1,0,0) & f(1,0,1) & f(1,1,1) & f(1,1,0) \\
\end{tabular}

\subsection{Примеры}

Для примеры я возьму элементарные функции выраженные через СДНФ и СКНФ.

\vspace{1em}

\textbf{Пример 1:} $\overline{x_{1}} \land x_{2} \lor  x_{1} \land {x_{2}} $


Заполненная карта карно для функции $f_{1}(x)$:

\vspace{1em}

\begin{NiceTabular}{c|c|c}[colortbl-like]
$x_{1} \backslash x_{2}$ & 0 & 1 \\
\hline
0 & 0 & 1 \\
1 & 0 & 1 \\
\CodeAfter
\tikz \node[draw=red, thick, ellipse, inner sep=2pt, fit=(2-3)(3-3)] {};
\end{NiceTabular}

\vspace{1em}

После чего создадим матрицу (1), в которой будут записанные все значения $x_{1} и x_{2}$, которым соответствует значение 1 определённой нами булевой функции.

\begin{tabular}{c|c}
$x_{1}$ &  $x_{2}$ \\
\hline
0 & 1 \\
1 & 1 \\
\end{tabular}
\newpage
Запишем элемент с неизменным значением - $x_{2}$. Тогда наша новая формула - просто $x_{2}$. Построим таблицу истинности.

\begin{tabular}{c|c|c}
$x_{1}$ & $x_{2}$ & $f_{2}(x_{2})$ \\
\hline
0 & 0 & 0 \\
0 & 1 & 1 \\
1 & 0 & 0 \\
1 & 1 & 1 \\
\end{tabular}

Как видно, значения функции $f_{2}$ повторяют значения функции $f_{1}$, но $f_{2}$ игнорирует 1-й аргумент.

\vspace{1em}

Это связанно с тем, что в нашей матрице (1) значения $x_{2}$ остаються неизменными, в то время как значения $x_{1}$ пробегают все возможные значения, а значит, независимо от того, какие значения принимает $x_{1}$, функция будет истинна пока $x_{2}$ принимает значение 1.

\vspace{1em}

В этом и заключаеться смысл карт карно. Матрица построенна таким образом, что объединения собирают в себе неизменность некоторых переменных при полном пробегании значений других. Это позволяет исключить ненужные переменные из формул.

\vspace{1em}

Рассмотрим более сложный пример, уже с тремя переменными:

\textbf{Пример 2:}($\overline{x_{1}} \land x_{2} \land x_{3}) \lor (\overline{x_{1}} \land x_{2} \land \overline{x_{3}}) \lor \overline{x_{1}} \land \overline{x_{2}} \land x_{3}) \lor (\overline{x_{1}} \land \overline{x_{2}} \land \overline{x_{3}}) \lor (x_{1} \land  \overline{x_{2}} \land x_{3}) \lor(x_{1} \land x_{2} \land x_{3})$

\vspace{1em}

Также строим матрицу (2.1):

\vspace{1em}

\begin{NiceTabular}{c|c|c|c|c}[colortbl-like]
$x_{1} \backslash x_{2}x_{3} $ & 00 & 01 & 11 & 10 \\
\hline
0 & 1 & 1 & 1 & 1 \\
1 & 0 & 1& 1 & 0 \\
\CodeAfter
\tikz \node[draw=red, thick, ellipse, inner sep=2pt, fit=(2-3)(3-3)(2-4)(3-4)] {};
\tikz \node[draw=blue, thick, ellipse, inner sep=2pt, fit=(2-2)(2-5)] {};
\tikz \node[draw=black, thick, ellipse, inner sep=2pt, fit=(2-2)] {};
\tikz \node[draw=black, thick, ellipse, inner sep=2pt, fit=(2-5)] {};
\end{NiceTabular}

\vspace{1em}

Помеченные черным элементы здесь также стоит принимать как объединённые, вполне позволительно представить матрицу как торус, и посему объединять элементы, как бы разорванные друг от друга. Воспринимать данную матрицу стоит как развёртку, так что знакомым с 3D моделированием людям тут будет слегка проще понять аналогию. Итак, наша задача разными способами выделить нужны нам единицы в фигуры, больше пересечений - меньше оптимизации и смысла в данной работе. Фигуры что мы можем использовать это квадрат, прямоугольник 1x2, и прямоугольник 1x4.

\vspace{1em}

Чтобы не запутаться, вот весь чётко сформулированный список требований, которым дожна удовлетворять группировка: 

1. Группы - прямоугольники с размером $2^{n} \times 2^{m}, n, m \in Z$

2. Размеры групп - $2^{k}, k \in Z$

3. Группы могут оборачиваться по краям таблицы, как будто она на торе.

4. Каждая 1-ца может входить в несколько групп

\newpage

В нашем случае я возьму синую и красную группы:

\begin{tabular}{c|c|c}
$x_{1}$ & $x_{2}$ & $x_{3}$ \\
\hline
0 & 0 & 1 \\
0 & 1 & 1 \\
1 & 0 & 1 \\
1 & 1 & 1 \\
\end{tabular}
\hspace{1cm}
\begin{tabular}{c|c|c}
$x_{1}$ & $x_{2}$ & $x_{3}$ \\
\hline
0 & 0 & 0 \\
0 & 0 & 1 \\
0 & 1 & 0 \\
0 & 1 & 1 \\
\end{tabular}

\vspace{1em}

Из первой таблицы - $\overline{x_{1}}$  (Если повторяються нули, то нужно взять отрицания данного аргумента).

Из второй таблицы - $x_{3}$

\vspace{1em}

Объедиие:  $\overline{x_{1}} \lor x_{3}$

\vspace{1em}

Как видите, наша длинная формула свернулась до такой вот короткой записи, осталось только это проверить:

\begin{tabular}{c|c|c|c}
$x_{1}$ & $x_{2}$ & $x_{3}$ & $f()$ \\
\hline
0 & 0 & 0 & 1 \\
0 & 0 & 1 & 1 \\
0 & 1 & 0 & 1 \\
0 & 1 & 1 & 1 \\
1 & 0 & 0 & 0 \\
1 & 0 & 1 & 1 \\
1 & 1 & 0 & 0\\
1 & 1 & 1 & 1 \\
\end{tabular}

\vspace{1em}

Всё совпало с матрицей (2.1), а значит мы получили верный ответ. На этом всё.

\section{Заключение}

В данной работе я объяснил и на премерах показал алгорит работы с картами карно, этого объяснения должно быть достаточно для примерного понимания их работы.

\end{document}